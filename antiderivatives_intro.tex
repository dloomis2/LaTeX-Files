\documentclass{article}
\usepackage{graphicx} % Required for inserting images

\title{antiderivatives and Differential Equations}
\author{David Loomis}
\date{April 2025}

\usepackage{setspace}
% Using \doublespacing in the preamble 
% changes the text to double-line spacing
\doublespacing

\begin{document}

\maketitle

\section{Antiderivatives}

\paragraph{Introduction}
Antiderivatives are the inverse operations of derivatives. In other words, taking the derivative of an antiderivative will give you the original function.\newline
An example of this is as follows:\newline
$\frac{dy}{dx}F(x)=f(x)$ , where $F(x)$ is the antiderivative of $f(x)$.\newline
More specifically, we can use a function such as $f(x)=2x^3+3x^2+2x$.\newline
To find the antiderivative of this function, we must think of what gives us $f(x)$ after taking the derivative.
The following is how to solve the problem.\newline


\paragraph{Example:}

Use the power rule for antiderivatives, defined as $x^n=\frac{1}{n+1}x^n+1$, where $n\neq -1$\newline
Applying it to $f(x) \Longrightarrow F(x)=\frac{1}{2}x^4 + x^3+x^2+C$\newline
Here $C$ refers to an arbitrary constant, since this is a general solution where $C$ could be any number $n$ $s.t.$ 
$n\in\mathbb{R}$.\newline

\paragraph{Starting with Derivatives:}
\newline 

Suppose we are given the following:\newline

Find an equation for $f(x)$ where $f''(x)=2sin(x)$, $f(0)=2$ and $f'(0)= \pi$\newline

Taking the antiderivative of $f''(x)$ gives us $f'(x)$\newline

$\Longrightarrow f'(x)=-2\cos(x) + C$\newline

Following the pattern, the antiderivative of $f'(x) = f(x)$\newline

$\Longrightarrow$ $f(x)=-2\sin(x)+Cx+D$\newline

\paragraph{Initial Values:}
We get another constant, $D$, from taking the derivative again. Exactly like $C$, $D\in\mathbb{R}$
\newline

Plugging in $0$ into $f'(x)$: $f'(0)=-2\cos(0)+C=\pi$\newline

$f'(0)=-2+C=\pi \Longrightarrow C=\pi+2$\newline

Plugging in $0$ into $f(x)$: $f(0)=-2\sin(0)+Cx+D=2$\newline

$f(0)=-2sin(0)-\pi(0)+D=-2 \Longrightarrow D=2$\newline

With these conditions met, we plug $C$ & $D$ back into $f(x)$:\newline

The final answer is $f(x)=-2\sin(x)+(-\pi+2)x+2$

\end{document}

